
\chapter{Modello Concettuale}
 	\label{cap:modello concettuale}

Il modello sviluppato \'e composto da un ramo principale (comprensivo di Front 
Server, Back-End Server e relative code) e da una componente di retroazione che 
si compone di un centro di Client, all'interno del quale gli utenti passano un 
certo tempo a pensare prima di effettuare la richiesta
successiva.

Al sopraggiungere di una nuova sessione, questa verr\`a processata 
immediatamente dal Front Server, nel caso in cui la sua coda sia vuota, 
altrimenti verr\`a posta in attesa del proprio turno con un conseguente ritardo. 
Una volta che il Front Server ha elaborato tale sessione, quest'ultima verr\`a 
processata all'interno di un Back-End Server, se la coda di quest'ultimo \'e 
vuota, altrimenti potrebbe subire un certo ritardo dovuto all'attesa del 
proprio turno. Al termine di tale servizio la sessione uscir\`a dal sistema, nel 
caso in cui abbia completato tutte le richieste che la componevano, altrimenti 
verr\`a reindirizzata nuovamente all'ingresso del sistema attraverso una ramo di 
feedback.

In tale ramo la sessione permane in un centro di Client in cui l'utente pu\`o 
spendere del tempo per pensare alla sua richiesta successiva. Dopo tale attesa 
la sessione tenta di rientrare nel sistema per ricevere un ulteriore servizio, 
andandosi a posizionare alla fine della coda FIFO del Front Server, inoltre tale 
servizio \'e senza prelazione ed \'e conservativo.

Il sistema verr\`a regolamentato attraverso un meccanismo di \textit{overload 
management} che permetter\`a di limitare i tempi di esecuzione della 
simulazione, nel caso della distribuzione peggiore. Questo sistema di gestione del carico 
consiste nel monitorare l'utilizzazione del Front Server, infatti appena 
raggiunto l' 85\%, il sistema rigetter\`a tutte le richieste (sia le nuove in arrivo 
sia quelle gi\`a presenti nel sistema), fino a che l'utilizzazione non raggiunga una soglia inferiore al 75\%. 
Tale tipologia di simulazione garantisce una notevole semplicit\`a di gestione 
dell'intero sistema attraverso la facilit\`a di avanzamento del tempo ed il 
controllo delle diverse tipologie di eventi che occorrono durante le varie 
esecuzioni.

\section{Variabili di Stato}
Il sistema \'e descritto completamente dalle seguenti variabili di stato:
\begin{itemize}
\item $busy_{fs}^{}$= stato di occupazione del Front Server
\item $queue\_length_{fs}^{}$= numero di richieste nella coda del Front Server
\item $busy\_{bes}^{}$= stato di occupazione nella coda del Back-End Server
\item $queue\_length_{bes}^{}$= numero di richieste nella coda del Back-End 
Server
\item $client\_counter$= numero di client attivi in un dato istante di tempo

\end{itemize}
Al fine di calcolare medie, varianze, intervalli di confidenza e per 
visualizzare l'avanzare della simulazione sono state utilizzate anche altre 
variabili (per lo più booleane o di tipo contatore): \textit{arrivals\, 
sessions\, requests\, dropped\, aborted\,} $…$ .

\section{Modello delle specifiche}
Nello sviluppo del modello delle specifiche, l'attenzione \'e stata rivolta alla 
definizione dei modelli di input da utilizzare nel modello di simulazione. Tali 
modelli sono stati definiti in base alle specifiche fornite nel seguente modo:

 \begin{itemize}
	\item $\lambda_{sessioni} = 35 \frac{richieste}{s}$ (distribuito 
esponenzialmente)
 	\item $Dimensione_{sessioni} \sim Equilikely(5, 35)$
 	\item $E[Z] = 7 s$ (distribuito esponenzialmente)
 	\item $E[D]_{front-end} = 0.00456 s$ (distribuito esponenzialmente)
 	\item $E[D]_{back-end} = 0.00117 s$ (distribuito esponenzialmente)
 \end{itemize} 

\section{Eventi}
Dopo aver prodotto un modello delle specifiche sono stati identificati gli 
eventi generati dal sistema, durante la simulazione:
\begin{itemize}
\item \textbf{NewSession}: una nuova sessione entra nel sistema. Ovvero verr\`a inserita 
nel Front Server o nella sua coda qualora quest'ultimo fosse gi\`a occupato;
\item \textbf{FS\_Completion}: il Front Server evade una richiesta e la invia al Back-End 
Server;
\item \textbf{BES\_Completion}: il Back-End Server evade una richiesta. La sessione 
corrispondente pu\`o dunque essere completata del tutto e quindi uscire dal 
sistema oppure migrare verso il centro Client nel caso in cui il suo numero di 
richieste sia non nullo.
\item \textbf{Client\_Completion}: dopo aver passato un certo tempo in fase di 
“\textit{Thinking}”, una sessione lascia il centro Client e rientra nel sistema 
attraverso il ramo di retroazione, quindi \'e direttamente inserita nella coda 
del Front Server.
\end{itemize}
