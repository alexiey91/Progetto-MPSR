\chapter{Modello Computazionale}
Il programma realizzato \'e composto da un file eseguibile (simulatore.c) e di 
alcuni file dove sono contenute funzioni di appoggio. (\textit{event\_list}, 
\textit{arrival\_queue}, \textit{autocorrelation}, \textit{client\_req}, 
\textit{req\_queue}).
Il software sviluppato \'e codificato con il linguaggio \textit{C}.

\section{Simulatore.c}

Questo applicativo \'e il cuore del simulatore implementato. Tale programma 
sfrutta un'interfaccia testuale per interrogare l'utente circa la 
configurazione 
da adottare per la simulazione da eseguire. 
\'E possibile selezionare diverse opzioni:
\begin{itemize}
\item Scegliere la distribuzione da testare;
\item Scegliere se attivare o meno il meccanismo di gestione dell'overload;
\item Scegliere il seed per la generazione di numeri random;
\item Scegliere i parametri della simulazione: numero e lunghezza dei batch %nel caso sia attivata la gestione dell'overload;
\begin{comment}
\item Scegliere i parametri della simulazione: Start e Stop nel caso non di assenza di overload management.
\end{comment}
\item Scegliere se visualizzare lo stato della simulazione live.
\end{itemize}

I risultati prodotti dalla simulazione vengono trascritti su un file di tipo 
"\textit{.csv}"  in modo da dare all'utente una chiara ed equilibrata visione 
dei dati ottenuti.

\section{Event List}
La lista di eventi \'e costituita da strutture di tipo \textit{Event}, formate 
da un campo di tipo \textit{double}, che rappresenta il tempo 
di occorrenza dell'evento, un \textit{\_EVENT\_TYPE} che identifica il tipo di 
evento 
ed un puntatore \textit{next} alla struttura seguente.
Per merito della funzione \textbf{add\_event()} \'e possibile aggiungere eventi 
alla lista. Verranno inseriti seguendo un ordinamento crescente rispetto alla 
variabile time. Durante l'inserimento dei dati viene effettuato un controllo 
sulla consistenza degli stessi, cio\'e si controlla, con una funzione 
\textbf{event\_check()}, che il tempo sia un valore positivo e che il tipo di 
evento appartenga all'insieme degli eventi noti (es: \texttt{nuova sessione}, 
\texttt{fs completition}, ...).     
Gli eventi vengono estratti dalla struttura attraverso la funzione 
\textbf{pop\_event()}, che preleva l'elemento in testa alla lista, 
restituendolo 
alla funzione chiamante.

\section{Arrival Queue}
Definisce una struttura dati, utilizzata per modellare le code sia del Front Server
che del Back-End Server. Ogni coda associata ad un centro si compone di un tempo di arrivo e un puntatore all'elemento successivo. 
Le funzioni messe a disposizione per questa struttura dati sono: 
l'\textit{arrival\_add()} che aggiunge un nuovo elemento posizionandolo in fondo alla coda, l'\textit{arrival\_pop()} 
che permette l'estrazione dell'elemento in testa e l'\textbf{arrival\_print()} che stampa lo stato della coda.

\section{Request Queue}
Il sistema all'arrivo di una nuova sessione genera in maniera random,
con distribuzione esponenziale, un numero di richieste comprese tra 5 e 35.
Per evitare di memorizzare tale informazione all'interno della sessione stessa
\'e stata creata, questa struttura dati adibita a raccogliere tali parametri.
Non appena la sessione arriva nel primo centro, ossia il Front Server, il dato 
contenente il numero di richieste associate, viene inserito in questa coda tramite la funzione 
\textbf{enqueue\_req()}. Quando invece la sessione esce dal Back-End Server è necessario
l'utilizzo della funzione di \textbf{dequeue\_req()} poichè questa informazione non verrà pi\`u
memorizzata in questa lista, ma verrà salvata in una diversa struttura, nel caso la sessione non sia completata.
Per verificarne la consisenza \'e stata implementata la funzione \textbf{print\_req()}.

\section{Client request}
Questa struttura dati ci viene in aiuto nel momento in cui la sessione 
esce dal Back-End Server per entrare nel centro di client. In questo caso 
infatti non vengono pi\`u utilizzate le strutture precedentemente descritte,
quali \textit{Request Queue} e \textit{Arrival Queue}, per modellare il comportamento 
del sistema, poichè durante la fase di ``thinking'' l'ordine di entrata delle sessioni 
non corrisponde a quello di uscita. Di conseguenza \'è stata creata questa struttura,
che al suo interno contiene sia il tempo di arrivo della sessione nel centro, sia il numero
di richieste rimanenti al suo completamento, per garantire una corretta corrispondenza tra la sessione ed il suo numero di richieste.
Le funzioni di cui dispone questa struttura, sono: 
\textbf{add\_client\_req()}, \textbf{pop\_ClientReq()} e 
\textbf{print\_client\_req()}, esattamente identiche a quelle delle strutture 
precedenti.

\section{File Manager}
La parte relativa al file manager gestisce tutto il flusso di dati da 
trascrivere  su un file di output. Composto da tre funzioni:
\begin{itemize}
\item \textbf{get\_date()}: permette di ottenere l'orario e la data correnti, 
da 
salvare sul file desiderato.
\item \textbf{open\_file()}: utilizzata per la creazione e l'apertura 
del file da salvare. Il nome del file viene creato utilizzato la funzione 
\textbf{get\_date()}, quindi contiene la data corrente della creazione più il 
tipo di 
distribuzione scelta durante la fase di setting.
\item \textbf{close\_file()}: adottata per chiudere il file una volta 
terminata la scrittura su di esso.
\end{itemize}

\section{Utils}
Il file \textbf{utils.h} contiene delle funzioni utilizzate per la pulizia della 
console 
e alcune funzioni per la manipolazione dei dati ottenuti al termine della 
simulazione.

\begin{comment}
\section{Autocorrelazione}
Al fine di calcolare l'autocorrelazione sui tempi di risposta del sistema con 
\textbf{LAG} pari a 20 \'e stato scritto un programma C che prende in input il 
file generato dal simulatore e restituisce i valori calcolati delle 
autocorrelazioni. Per il calcolo di queste si \'e usata la formula:

\vspace{0.5cm}
\begin{center} $r_{j} = \frac{c_{j}}{c_{0}} con j = 1,2,...20$\end{center}

\section{Intervalli di confidenza}
\'E stato sviluppato un programma scritto in linguaggio \textbf{C} allo scopo 
di 
calcolare gli intervalli di confidenza di livello 1 - \textalpha, dove 
\textalpha è pari al 5\%. Tale programma prende in input un file in cui sono 
scritti tutti i valori su cui si vuole fare il calcolo e computa le seguenti 
statistiche:
\begin{itemize}
 \item Calcolo della media: $\bar{x}_{i} = \frac{1}{i}(x_{i} - \bar{x}_{i-1})$;
 \item Calcolo della varianza: $v_{i} = v_{i-1} + (\frac{i-1}{i}){(x_{i} - 
\bar{x}_{i-1})}^{2}$;
 \item Calcolo del valore critico: $t^{*} = idfStudent(n-1, 
1-\frac{\textalpha}{2})$;
 \item Calcolo degli estremi dell'intervallo: $\bar{x} \pm 
\frac{t^{*}s}{\sqrt{n-1}}$;
\end{itemize}
\end{comment}