\chapter{Modello Computazionale}
Il programma realizzato è composto da un file eseguibile (simulatore.c) e di alcuni file dove sono contenute funzioni di appoggio. (\textit{event\_list}, \textit{arrival\_queue}, \textit{autocorrelation}, \textit{client\_req}, \textit{req\_queue}).
Il software sviluppato è codificato con il linguaggio \textit{C}.

\section{Simulatore.c}

Questo applicativo è il cuore del simulatore implementato. Tale programma sfrutta un'interfaccia testuale per interrogare l'utente circa la configurazione da adottare per la simulazione da eseguire. 
È possibile selezionare diverse opzioni:
\begin{itemize}
\item Scegliere la distribuzione da testare tra:
\begin{itemize}
\item \textit{Esponenziale}
\item \textit{HyperEsponenziale}
\item \textit{10-Erlang}
\end{itemize}
\item Scegliere il seed per la generazione di numeri random
\begin{itemize}
\item 615425336
\item 37524306
\item 123456789
\end{itemize}
\item Scegliere i parametri della simulazione:
\begin{itemize}
\item \textit{Primo Stop}
\item \textit{Ultimo Stop}
\item \textit{Numero di esecuzioni}
\end{itemize}
\item La possibilità di stampare sulla console le variabili più importanti durante la simulazione.
\end{itemize}

Alla luce di quanto illustrato in precedenza si può sottolineare la capacità del programma di poter calcolare automaticamente il passo della simulazione: l'utente dovrà soltanto inserire il parametro iniziale e finale della simulazione ed il numero di run che desidera eseguire, il software genererà il passo unitario per il ciclo di simulazione.
Inoltre all'interno del programma è stato implementato un meccanismo di \textit{Overload Management}, che l'utente può abilitare o disabilitare, applicato poi nella distribuzione della \textit{10-Erlang}.
I risultati prodotti da questa simulazione vengono trascritti su un file di tipo "\textit{.csv}"  in modo da dare all'utente una chiara ed equilibrata visione dei dati ottenuti.

\section{Event List}
La lista di eventi è costituita da strutture di tipo \textit{Event}, formate da un campo di tipo \textit{double}, che indica il time che rappresenta il tempo di occorrenza, un \textit{\_EVENT\_TYPE} type rappresentante il tipo di evento ed un puntatore \textit{next} alla struttura seguente.
Per merito della funzione \textit{add\_event()} è possibile aggiungere eventi alla lista. Verranno inseriti seguendo un ordinamento crescente rispetto alla variabile time. Durante l'inserimento dei dati viene effettuato un controllo sulla consistenza dei dati, cioè: si controlla, con una funzione \textit{event\_check()}, che il tempo sia un valore positivo, che il tipo di evento sia compreso tra 0 e 3.
Gli eventi vengono estratti dalla struttura attraverso la funzione \textit{pop\_event()}, che preleva l'evento in testa alla lista, restituendolo alla funzione chiamante.

\section{Arrival queue}
La coda di arrivi contiene la struttura di dati di riferimento, che permette la gestione degli arrivi nei vari centri quali: \textit{Front Server}, \textit{Back-End Server}, \textit{Centro Client}. La coda viene rappresentata con una semplice struttura dati formata da: un tempo di arrivo e un puntatore all'elemento successivo. Le funzioni messe a disposizione per questa struttura di dati sono: \textit{arrival\_add()}, \textit{arrival\_pop()} e l'\textit{arrival\_print()}.
\begin{itemize}
\item \textit{arrival\_add()}: genera un nuovo elemento contenente il tempo di arrivo in un determinato centro,scorre tutto il contenuto della lista generata, posizionando il nuovo nodo in fondo alla coda.
\item \textit{arrival\_pop()}: permette l'estrazione dell'elemento meno recente della coda.
\item \textit{arrival\_print()}: stampa lo stato della coda.
\end{itemize}

\section{Request queue}
La coda delle richieste si basa sul concetto di richiesta: ogni sessione, una volta ammessa all'interno del sistema, definisce un numero di richieste compreso tra 5 e 35. Questa informazione viene inserita all'interno della coda delle richieste in modo da poter sfruttare i dati generati, per modellare gli utenti attivi durante la simulazione.
Sfruttando la funzione \textit{enqueue\_req()} è possibile inserire tutte le richieste generate dalla nuova sessione vigente. Per poter rimuovere tale dato è possibile  utilizzare la \textit{dequeue\_req()}.
Per verificare l'andamento di tale struttura dati è stata implementata la funzione \textit{print\_req()}.

\section{Client request}
La client\_req.h implementa una struttura dati in grado di propagare l'informazione circa il numero di richieste relative ad una sessione. Questa viene impiegata all'ingresso e all'uscita dal centro di client in quanto l'ordine di entrata all'interno della zona del \textit{Think Time} è differente da quello di uscita.
Per quanto riguarda le funzioni, implementate all'interno di questo programma hanno le stesse funzionalità espresse nelle sezioni precedenti, sono: \textit{add\_client\_req()}, \textit{pop\_ClientReq()}, \textit{print\_client\_req()}.

\section{File Manager}
La parte relativa al file manager gestisce tutto il flusso di dati da trascrivere  su un file di formato "\textit{.csv}". Composto da tre funzioni:
\begin{itemize}
\item \textit{get\_date()}: permette di ottenere l'orario e la data correnti, da salvare sul file desiderato.
\item \textit{open\_file()}: funzione utilizzata per la creazione e l'apertura del file da salvare. Il nome del file viene elaborato utilizzato la funzione \textbf{get\_date()}, quindi con la data corrente della creazione più il tipo di distribuzione scelta durante la fase di setting.
\item \textit{close\_file()}: funzione adottata per chiudere il file una volta terminata la scrittura su di esso.
\end{itemize}

\section{Utils}
Il file utils.h contiene delle funzioni utilizzate per la pulizia della console e alcune funzioni di scrittura su file di tipo \textit{.csv}, circa i dati ottenuti al termine di una simulazione completata.

\section{Salvataggio dei dati}
Il programma "\textbf{simulatore}" serializza tutta le informazioni, generate durante la simulazione, utili al calcolo delle grandezze medie su un file, inserendo blocchi di dati al termine di ogni run.
Il nome dei file è legato alla data ed all'orario di avvio della simulazione e al tipo di distribuzione adottata.
Il tipo di file che viene generato è di tipo \textit{.xls}. Questi sono adibiti alla creazione automatica dei grafici. Il programma quindi generata automaticamente un file con delimitatori di cella e di riga al fine di consentire all'utente una facile consultazione dei dati ottenuti o semplicemente una facile creazione dei diagrammi. Questi file verranno adottati durante la fase di analisi dei risultati.

\section{Autocorrelazione}
Al fine di calcolare l'autocorrelazione sui tempi di risposta del sistema con \textbf{LAG} pari a 20 è stato scritto un programma C che prende in input il file generato dal simulatore e restituisce i valori calcolati delle autocorrelazioni. Per il calcolo di queste si è usata la formula:

\vspace{0.5cm}
\begin{center} $r_{j} = \frac{c_{j}}{c_{0}} con j = 1,2,...20$\end{center}

\section{Intervalli di confidenza}
\'E stato sviluppato un programma scritto in linguaggio \textbf{C} allo scopo di calcolare gli intervalli di confidenza di livello 1 - \textalpha, dove \textalpha è pari al 5\%. Tale programma prende in input un file in cui sono scritti tutti i valori su cui si vuole fare il calcolo e computa le segueni statistiche:
\begin{itemize}
 \item Calcolo della media: $\bar{x}_{i} = \frac{1}{i}(x_{i} - \bar{x}_{i-1})$;
 \item Calcolo della varianza: $v_{i} = v_{i-1} + (\frac{i-1}{i}){(x_{i} - \bar{x}_{i-1})}^{2}$;
 \item Calcolo del valore critico: $t^{*} = idfStudent(n-1, 1-\frac{\textalpha}{2})$;
 \item Calcolo degli estremi dell'intervallo: $\bar{x} \pm \frac{t^{*}s}{\sqrt{n-1}}$;
\end{itemize}