\chapter{Obiettivi}
 	\label{cap:obiettivi}
 L'obiettivo del progetto è quello di modellare, pianificare e sviluppare un simulatore di traffico web che rispetti le specifiche di consegna.
 
\vspace{0.5 cm} \noindent Il sistema reale presenta le seguenti caratteristiche:
 \begin{itemize}
 \item un flusso di utenti che si connettono al sistema sotto forma di sessioni;
 \item un numero di richieste di cui si compone ogni sessione;
 \item un front-end server;
 \item un back-end server con un database.
 \end{itemize}

 Il simulatore verrà utilizzato per analizzare il comportamento stazionario relativo ad alcuni indici prestazionali quali il tempo di risposta del sistema, il throughput e la percentuale di sessioni abortite e rifiutate.
 Il sistema dispone infatti anche di un meccanismo di \emph{gestione del sovraccarico} basato sul monitoraggio in tempo reale dell'utilizzazione del front-end.\\

 Le specifiche fornite dalla traccia sono le seguenti:
 \begin{itemize}
 	\item $\lambda_{sessioni} = 35 \frac{richieste}{s}$ (distribuito esponenzialmente)
 	\item $Dimensione_{sessioni} \sim Equilikely(5, 35)$
 	\item $E[Z] = 7 s$ (distribuito esponenzialmente)
 	\item $E[D]_{front-end} = 0.00456 s$ (distribuito esponenzialmente)
 	\item $E[D]_{back-end} = 0.00117 s$ (distribuito esponenzialmente)
 \end{itemize} 
