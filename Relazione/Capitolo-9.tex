\chapter{Analisi dei Risultati} 

I dati relativi agli indici di prestazione di interesse (tempi di risposta e throughput) misurati 
nelle precedenti fasi di test sono stati in seguito aggregati e visualizzati in forma di grafici. 
Infatti, in base al tipo di test effettuato, tali grafici si suddividono in due categorie: i grafici di 
throughput e tempi di risposta del sistema in stato di instabilit\`a (front server sovraccarico) e 
quelli relativi al sistema in condizione di stazionariet\`a (front server non sovraccarico), ovvero quelli gesiti
attraverso il meccanismo di overload management.

\section{Sistema senza Overload Management}

\subsection{Response Time}

In questo scenario l’utilizzazione del front-end server \'e praticamente uguale a 1, perci\`o 
non riesce a completare tutte le richieste entranti, con la inevitabile 
situazione di veder aumentare indefinitamente la lunghezza della sua coda. Di 
conseguenza il tempo di risposta del sistema tende a divergere all’aumentare del tempo di 
simulazione. Il grafico illustrato di seguito mostra tale scenario:

\begin{figure}[H]
 \centering
% \includegraphics[scale=0.7]{img/}
 \caption[Tempi di risposta del sistema instabile]{Tempi di risposta del sistema instabile}
 \label{fig:Tempi di risposta del sistema instabile}
\end{figure}


\subsection{Autocorrelazione}

\subsection{Useful Throughtput}

\section{Sistema con Overload Management}

\subsection{Response Time}

\subsection{Autocorrelazione}

\subsection{Useful Throughtput}

\subsection{Drop e Aborted Ratio}