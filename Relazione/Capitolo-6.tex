\chapter{Verifica}

La fase di verifica \'e molto importante poich\'e consente di dimostrare la 
consistenza del programma creato con il modello delle specifiche.

\vspace{0.3cm}In primis, \'e stata utilizzata una funzione di stampa per 
verificare il corretto flusso delle sessioni all'interno dell'intero sistema. 
Tale funzione, la \texttt{print\_system\_state()}, stampa le statistiche pi\`u 
importanti del sistema in tempo reale su standard output.

\vspace{0.3cm}In secundis, si \'e notato e verificato che le liste contenenti le 
sessioni e le richieste si riempiono e si svuotano in modo corretto. Anche in 
questi casi sono state necessarie funzioni di stampa per verificare che 
l'aggiornamento fosse adeguato.

\vspace{0.3cm}In terzis, i vincoli sullo stato del sistema e sull'entrata in 
azione dell'overload manager, e sul calcolo delle medie sono tutti soddisfatti.

\vspace{0.3cm}Il simulatore parte con il numero di sessioni nullo e tutte le 
variabili di stato e di supporto sono inizializzate opportunamente. Il 
sistema consente di avere una stampa aggiornata di tutti i parametri 
rilevanti, come throughput e tempo medio di risposta.

\vspace{0.3cm}Il numero di sessioni rifiutate e abortite cresce in maniera 
consistente con il meccanismo di controllo delle ammissioni.

\vspace{0.3cm}Infine, come ultima verifica, \'e stato dimostrato che superato il 
numero massimo di sessioni per batch, che l'utente ha inserito manualmente, il sistema passa a simulare il batch
successivo, e una volta raggiunto il numero massimo di batch, impostati sempre dall'utente, il programma termina
correttamente la propria esecuzione.

