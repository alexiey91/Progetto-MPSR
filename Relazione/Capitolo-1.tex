\chapter{Generatore di Lehmer}
	\label{cap:generatorerandom}

Il progetto si pone di testare il noto generatore pseudo-casuale di numeri 
random di Lehmer utilizzando, a tale scopo, uno dei test di casualit\'a 
illustrati nel libro \textit{Discrete-Event Simulation: A first course}.
All'interno del progetto si \'e scelto di utilizzare la versione del generatore
fornita come libreria C dallo stesso libro, \'e stato quindi creato un programma che si interfaccia con tali librerie, tramite chiamate alle API della libreria 
stessa,
che serve a testare l'effettiva correttezza di tale implementazione.
Al fine di comprendere al meglio i risultati ottenuti, verranno anche
presentati dei grafici riassuntivi del test effettuato.

\section{Funzionamento}
Il \textit{generatore di Lehmer} \'e un generatore basato su un algoritmo che d\`a 
origine ad una sequenza di numeri pseudo-casuali. \'E definito da due parametri:

\begin{itemize}
 \item un modulo \textbf{\textit{m}}, che \'e un numero primo molto grande (in 
questo caso la libreria usa $2^{31} - 1$);
 \item un moltiplicatore \textbf{\textit{a}} che rappresenta un numero intero 
compreso tra 1 ed $m-1$.
\end{itemize}

\noindent La sequenza numerica pseudo-random viene generata tramite la formula :

\begin{center} $x_{i+1} = ax_{i} mod m$ \end{center}

\noindent Questa sequenza parte da un numero $x_{0}$ detto \textbf{seed}, anch'esso 
scelto tra 1 ed $m-1$. Non tutte le combinazioni di a ed m per\'o sono ottimali 
per realizzare una sequenza di numeri che garantiscano una buona randomicit\`a. 
Per verificare dunque se un seed e un moltiplicatore garantiscono un buon 
livello di 
randomicit\`a esistono dei test empirici. Nel paragrafo successivo tale generatore 
verr\`a sottoposto ad uno di questi.

\section{Test degli estremi}
Il test scelto per effettuare la verifica sul generatore, con parametri:
\begin{center}
$(a,m) = (48271, 2^{31} - 1)$
\end{center}

\noindent \'e conosciuto come ``\textit{Test degli estremi}''. 
Per la simulazione di tale test si pu\`o riassumere il processo in tre passi:
\begin{itemize}
 \item Generazione di un campione di valori con chiamate ripetute al generatore.
 \item Computazione di un test statistico la cui distribuzione (pdf o funzione 
di densit\`a di probabilit\`a) \'e nota su variabili random uniformi in (0,1) indipendenti e identicamente 
distribuiti.
 \item Valutare la verosimiglianza del valore computato del test statistico con 
la relativa distribuzione
 teorica da cui \'e stato assunto adottando una metrica basata sulla distanza 
lineare.
\end{itemize}

Questo test si basa sulla seguente considerazione:

\vspace{0.5cm} \noindent \textbf{Teorema} Se $U_{0}^{}$, 
$U_{1}^{}$,...,$U_{d-1}^{}$ \'e una sequenza di variabili \textit{Uniform(0,1)} e 
se 

\begin{center}R = max{$U_{0}^{}$, $U_{1}^{}$,...,$U_{d-1}^{}$} \end{center} 

\noindent allora la variabile U = $R_{}^{d}$ \'e una \textit{Uniform(0,1)} 
\footnote{Attenzione: il teorema afferma che la variabile U \'e una Uniform(0,1), 
mentre la variabile R non lo \'e.}.

In pratica tale test verifica che la variabilit\`a delle altezze dell'istogramma 
prodotto dai numeri pseudo-casuali generati \'e sufficientemente piccola da poter 
concludere che questi numeri appartengano ad una popolazione 
\textit{Uniform(0,1)}. Questa operazione \'e fondamentale in quanto tutte le 
altre distribuzioni di probabilit\`a contenute all'interno della libreria 
utilizzata vengono generate a partire dalla \textit{Uniform(0,1)}.

\section{Algoritmo}
L'algoritmo del test degli estremi effettua un raggruppamento (\textit{batching}) dei 
valori estratti dal generatore in gruppi di uguale lunghezza (determinato dal 
parametro \textit{d}), trovando il 
massimo di ogni batch, elevando tale massimo all d-esima potenza e conteggiando 
tutti i massimi generati in un array. Tale funzione viene applicata per ogni 
stream del generatore di Lehmer.
Tutto ci\`o \'e illustrato di seguito:
\begin{verbatim}
  for(stream = 0; stream < 256L; stream++) {
      long o[K];
      memset(o, 0, K * sizeof(long));
      for(i = 0; i < N; i++) {
          double r = Random();
          for(j = 1; j < D; j++) {
              u = Random();
              if(u > r) r = u;
          }
          u = exp(D * log(r));
          x = u * K;
          o[x] ++;
      }
  }
\end{verbatim}
Per ogni stream, si determina quindi la variabile chi-quadro $v$ tramite questa 
funzione:
\begin{verbatim}
      for(i = 0; i < K; i++) {
	      v += (square(o[i] - e_x));
      }
      v /= e_x;
\end{verbatim}

I valori critici $v_1^{*}$ e $v_2^{*}$ vengono calcolati utilizzando la funzione inversa 
\textit{idfChisquare(long n, double u)}  fornita dalla libreria rvms.c del libro di 
testo. Bisogna precisare che per il calcolo 
di tali variabili statistiche \'e stato scelto un livello di confidenza con 
parametro $\alpha$ = 0.05, mentre i parametri N e K sono rispettivamente N = 100000 
e K = N/20 = 5000.
Successivamente si confronta la statistica chi-quadro \textit{v}, determinata al passo 
precedente, con i valori critici  $v_1^{*}$ e $v_2^{*}$ , per ogni stream (in totale sono 256 
variabili chi-quadro $v$ ).
Se $v < v_1^{*}$ o $v > v_2^{*}$ il test fallisce (per quello stream) con probabilit\'a $1 - 
\alpha$.

\section{Test e conclusioni}
\noindent Il grafico risultante di questo test empirico \'e illustrato di seguito:

\begin{figure}[H]
 \begin{center}
  \includegraphics[scale=0.45]{img/test.png}
  \caption[Test degli estremi]{Test degli estremi}
  \end{center}
\end{figure}

\vspace{0.5cm}
I valori critici sono visualizzati come linee orizzontali: quella inferiore rappresenta
$v_1^{*}$ = 4804.92 mentre quella superiore è $v_2^{*}$ = 5196.86.

Dalla simulazione effettuata si \'e notato che il numero di test statistici $v > v_2^{*}$ sono stati esattamente 7, come il numero di test $v < v_1^{*}$.

\vspace{0.5cm}
Di seguito \'e riportato l'output del programma:

\begin{figure}[H]
  \begin{center}
  \includegraphics[scale=0.7]{img/test_estremi.png}
  \caption[Risultati]{Risultati}
  \end{center}
\end{figure}


Considerando il numero totale di test falliti (\textit{upper} e \textit{lower bound}) pari a 14, si nota che non
ci si discosta molto rispetto al valore atteso approssimato; infatti in 256 test con un livello di
confidenza del 95\% il valore atteso è circa 256 * 0.05 = 13 fallimenti. Questo valore pu\`o 
essere una indicazione della bont\`a del generatore di Lehmer implementato.
