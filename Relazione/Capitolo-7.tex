\chapter{Validazione} 

In questa fase viene testata la corrispondenza del simulatore con il modello reale. Sono stati effettuati i seguenti controlli:

\begin{itemize}
 \item Saturazione del Front Server: ci sono due concause che portano alla saturazione del Front Server, in primo luogo c'è una enorme differenza tra il tasso di servizio di quest'ultimo e quello del Back-End Server che è circa 3 volte maggiore; inoltre il numero di utenti che stazionano nel centro Client tende a crescere enormemente e tale comportamento influenza l'andamento di richieste in entrata al Front Server che ne determina la congestione.
 
 \item La coda del Back-End Server risulta essere quasi sempre vuota poichè è il Front Server che elabora in maniera lenta mentre il tasso di servizio in questo caso è di oltre 800 richieste / secondo; questo non permette la creazione di coda e di conseguenza l'utilizzazione è molto bassa.
 
 \item Il numero di client è notevolmente elevato: questo è dovuto all'elevato Think Time sperimentato dagli utenti, circa 7 secondi a client. Considerando il throughput del sistema ed i tassi di servizio dei due serventi è immediato notare come il numero dei client attivi contemporaneamente tenda a crescere nel tempo. Tale componente è modellato come un Infinite Server.
 
 \item Nel caso di overload management la percentuale delle connesioni rifiutate e abortite è molto alta arrivando a toccare anche l'87\% nel secondo caso, mentre nel primo è più contenuta, con un massimo che si attesta circa attorno al 30\% per poi oscillare in quell'intorno. Questo perchè il Front Server rappresenta il collo di bottiglia dell'intero sistema.
\end{itemize}