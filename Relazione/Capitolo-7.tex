\chapter{Validazione} 

\section{Modello Analitico}
Al fine di prevedere, in linea di massima, i risultati del simulatore, viene 
elaborato un modello  analitico semplificato per poter studiare il sistema preso 
in esame.
Il sistema implementato presenta numerosi vincoli ed una complessit\`a insita 
nelle specifiche, a tal proposito viene proposto un modello volontariamente e 
lievemente differente dal caso reale.

Il sistema esaminato risulta essere interattivo, in quanto avviene uno scambio 
di richieste tra Client e Server, con un comportamento a rete aperta (le nuove 
sessioni possono entrare nel sistema qualora questo non fosse saturo).
Questo modello permette di descrivere in dettaglio 
la maggior parte degli elementi che costituiscono il sistema stesso.

\section{Modello semplificato a rete aperta}
\begin{figure}[H]
  \centering
  \includegraphics[scale=0.7]{img/reteJackson.png}
  \caption[Modello a rete aperta]{Modello a rete aperta}
  \label{fig:Modello a rete aperta}
\end{figure}

In questo primo modello si ha una rete aperta di Jackson con un tasso di 
sessioni in arrivo pari a $\lambda$ . Si \'e deciso cos\`i, per questioni di 
semplificazione del modello, di non inserire il centro di Client, in quanto
risultavano di difficile analisi in un modello a rete aperta come quello di 
Jackson.
\begin{center}
$\lambda = 35 sessioni/s$
\\ \vspace{0.5cm}
$\begin{cases} 
y_{1} = \lambda + p y_{2} \\ y_{2} = y_{1} \\
\end{cases}$  $\rightarrow$
$\begin{cases} 
y_{1}(1-p) = \lambda \\ y_{2} = y_{1} \\
\end{cases}$ $\rightarrow$
$\begin{cases} 
y_{1} =\frac{ \lambda}{(1- p)} \\ y_{2} =\frac{\lambda}{(1-p)} \\
\end{cases}$
\\ \vspace{0.5cm}
$y_{1} = y_{2} = \frac{\lambda}{(1-p)} ; p=\frac{19}{20} \rightarrow 
y_{1} = y_{2} = 35\times20 = 700$ richieste/s
\\ \vspace{0.5cm}
$y_{1}\gg\mu_{FS}$; $y_{2}=\mu_{FS}$ poich\`e $\mu_{BES}\gg y_{1}$
\end{center}Viene quindi calcolato il throughput, ovvero il numero di sessioni che escono 
dal sistema al secondo:
\begin{center}
$y_{2} =\frac{\lambda}{(1-p)} \rightarrow \lambda=y_{2}(1-p)=\frac{\mu_{FS}}{20}\approx\frac{219}{20}$ richieste/s = $X_{sessioni}$
\end{center}
Per quanto concerne l'indice riguardante la percentuale di sessioni rifiutate 
dal sistema, si trova:
\begin{center}
 $dropped=\frac{\#sessioni accettate}{\#totale arrivi}$ $\rightarrow$ 
$\frac{35-X_{sessioni}}{35} = 1-\frac{2}{7}\approx 0.7 \approx 70\%$
\end{center}
\begin{comment}
\section{Modello semplificato chiuso}
\begin{center}	
	\begin{figure}[H]
	\centering
	\includegraphics[scale=0.7]{img/retechiusa.png}
	\caption[Modello a rete chiusa]{Modello a rete chiusa}
	\label{fig:Modello a rete aperta}
	\end{figure}
\end{center}

$D_{FS}=0.00456; D_{BES}=0.00117; E[Z]=7s ; N=250$ (in realt\'a il numero 
aumenta costantemente!).
$\vspace{2cm}\hspace{2mm}$
$\vspace{2mm}\hspace{2mm}$ Anche qui si pu\'o calcolare il throughput del 
sistema, come:
$\vspace{2mm}$
$X = min\{\frac{1}{D_{max}},\frac{N}{D+Z}\}= 
min\{\frac{1}{D_{FS}},\frac{250}{D_{FS}+D_{BES}+Z}\}=35.685 richieste/s$.
$\vspace{2mm}$
Da cui si pu\'o ricavare il lower bound per il tempo medio di risposta.
$\vspace{2mm}$
$E[R]\geq max\{D,\frac{N}{X}-E[Z]\}= max 
\{0.00573,\frac{250}{35}-7\}\approx0.0057447$
\end{comment}