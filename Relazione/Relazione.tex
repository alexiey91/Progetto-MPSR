 \documentclass[12pt, a4paper, openany, twoside, titlepage]{book}
 %opzione "twoside" anziché "oneside" per la stampa fronte-retro, "openrigth" anziché "openany" per far apparire i capitoli sempre a destra
 \usepackage[italian]{babel}
 \usepackage{fontenc}
 \usepackage[tc]{titlepic}	%gestisce l'immagine nel frontespizio
 %\usepackage[utf8]{inputenc}
 \usepackage{indentfirst}	%indenta le prime righe
 \usepackage{fancyhdr}		%gestisce note a margine, note a piè di pagina, e intestazioni
 \usepackage[pagebackref]{hyperref}	%gestisce i collegamenti ipertestuali
 \usepackage[scaled=0.95]{helvet}\selectfont		%utilizza il font selezionato per tutto il documento
 \usepackage{graphicx}		 %include le figure
 \usepackage{float}			 %package per il posizionamento delle figure
 \usepackage{amsmath}		 %include l'ambiente matematico
 \usepackage{enumitem}		 %include l'ambiente per le enumerazioni
 \usepackage{varwidth}		 %permette di centrare le enumerazioni
 \usepackage[nottoc]{tocbibind}	%gestisce il sommario
 \usepackage{enumitem}
 %permette di riprendere il conto di una enumerazione dove era stato interrotto
 %\usepackage{clrscode3e}
 %permette di scrivere pseudocodice (il file relativo si trova nella stessa cartella di questo sorgente Latex)
 \usepackage{subfigure}		 %permette di inserire sottofigure
 \usepackage{amsfonts}		 %gestisce il font matematico
 \usepackage{amssymb}		 %gestisce i simboli matematici
 %\usepackage{epigraph}	     %per la scrittura delle epigrafi a inizio capitolo
 %\usepackage{midpage}		 %per scrivere al centro della pagina
 %\usepackage{extsizes}		 %per utilizzare il tipo 'extbook'
 \usepackage{listings}		 %per inserire codice sorgente nel documento
 \usepackage{color}			 %per la gestione dei colori
 \usepackage{pifont}			 %introduce alcuni simboli
 \usepackage{moresize}		 %introduce nuove dimensioni per il testo
 \usepackage{pdfpages}
 \usepackage{rotating}
 \usepackage{comment}
 \usepackage{textgreek}
 \pagestyle{fancy}
 %\fancyhf{\small\thepage}
 \fancyhead[RO, LE]{}		 %solo i titotli dei capitoli in alto
 %\fancyhead[RO, LE]{\small\thepage}
 %\fancyhead[RE]{\small\nouppercase{\leftmark}}
 %\fancyhead[RO,LE]{\small\thepage}

 \def \openquote{``}		    %alias per aprire le virgolette
 \def \closequote{''}		%alias per chiudere le virgolette

 \def \abs#1{\lvert #1 \rvert}	 %alias per il simbolo di valore assoluto
 \def \quote#1{\openquote #1\closequote}		  %alias per il testo "virgolettato"

 \newcommand{\accg}{\`}		%per i caratteri con accento grave
 \newcommand{\acca}{\'}		%per i caratteri con accento acuto

 \begin{document}
 	\begin{titlepage}
 		\title{Modelli di Prestazioni di Sistemi  e Reti \\ \vspace{2 mm} {\large \bf{Simulatore di traffico in un sistema ``multi-tier''}}}
 		\author{Simone Martucci - \url{simone.martucci.91@gmail.com} \\ Alessandro Valenti - \url{alessandro.valenti1991@gmail.com} }
 		\date{}
 		\titlepic{\includegraphics[width=5cm]{img/logo_tor_vergata.png}}

 	\end{titlepage}

 	\maketitle

 	\frontmatter

 	\small

 	\pagenumbering{Roman}

 	\phantomsection

 	\addcontentsline{toc}{chapter}{Indice}

 	\tableofcontents
 	\label{Indice}

 	\normalsize

 	\mainmatter

 	\chapter{Generatore di Lehmer}
	\label{cap:generatorerandom}

Il progetto si pone di testare il noto generatore pseudo-casuale di numeri 
random di Lehmer utilizzando, a tale scopo, uno dei test di casualit\'a 
illustrati nel libro \textit{Discrete-Event Simulation: A first course}.
All'interno del progetto si \'e scelto di utilizzare la versione del generatore
fornita come libreria C dallo stesso libro, \'e stato quindi creato un programma che si interfaccia con tali librerie, tramite chiamate alle API della libreria 
stessa,
che serve a testare l'effettiva correttezza di tale implementazione.
Al fine di comprendere al meglio i risultati ottenuti, verranno anche
presentati dei grafici riassuntivi del test effettuato.

\section{Funzionamento}
Il \textit{generatore di Lehmer} \'e un generatore basato su un algoritmo che d\`a 
origine ad una sequenza di numeri pseudo-casuali. \'E definito da due parametri:

\begin{itemize}
 \item un modulo \textbf{\textit{m}}, che \'e un numero primo molto grande (in 
questo caso la libreria usa $2^{31} - 1$);
 \item un moltiplicatore \textbf{\textit{a}} che rappresenta un numero intero 
compreso tra 1 ed $m-1$.
\end{itemize}

\noindent La sequenza numerica pseudo-random viene generata tramite la formula :

\begin{center} $x_{i+1} = ax_{i} mod m$ \end{center}

\noindent Questa sequenza parte da un numero $x_{0}$ detto \textbf{seed}, anch'esso 
scelto tra 1 ed $m-1$. Non tutte le combinazioni di a ed m per\'o sono ottimali 
per realizzare una sequenza di numeri che garantiscano una buona randomicit\`a. 
Per verificare dunque se un seed e un moltiplicatore garantiscono un buon 
livello di 
randomicit\`a esistono dei test empirici. Nel paragrafo successivo tale generatore 
verr\`a sottoposto ad uno di questi.

\section{Test degli estremi}
Il test scelto per effettuare la verifica sul generatore, con parametri:
\begin{center}
$(a,m) = (48271, 2^{31} - 1)$
\end{center}

\noindent \'e conosciuto come ``\textit{Test degli estremi}''. 
Per la simulazione di tale test si pu\`o riassumere il processo in tre passi:
\begin{itemize}
 \item Generazione di un campione di valori con chiamate ripetute al generatore.
 \item Computazione di un test statistico la cui distribuzione (pdf o funzione 
di densit\`a di probabilit\`a) \'e nota su variabili random uniformi in (0,1) indipendenti e identicamente 
distribuiti.
 \item Valutare la verosimiglianza del valore computato del test statistico con 
la relativa distribuzione
 teorica da cui \'e stato assunto adottando una metrica basata sulla distanza 
lineare.
\end{itemize}

Questo test si basa sulla seguente considerazione:

\vspace{0.5cm} \noindent \textbf{Teorema} Se $U_{0}^{}$, 
$U_{1}^{}$,...,$U_{d-1}^{}$ \'e una sequenza di variabili \textit{Uniform(0,1)} e 
se 

\begin{center}R = max{$U_{0}^{}$, $U_{1}^{}$,...,$U_{d-1}^{}$} \end{center} 

\noindent allora la variabile U = $R_{}^{d}$ \'e una \textit{Uniform(0,1)} 
\footnote{Attenzione: il teorema afferma che la variabile U \'e una Uniform(0,1), 
mentre la variabile R non lo \'e.}.

In pratica tale test verifica che la variabilit\`a delle altezze dell'istogramma 
prodotto dai numeri pseudo-casuali generati \'e sufficientemente piccola da poter 
concludere che questi numeri appartengano ad una popolazione 
\textit{Uniform(0,1)}. Questa operazione \'e fondamentale in quanto tutte le 
altre distribuzioni di probabilit\`a contenute all'interno della libreria 
utilizzata vengono generate a partire dalla \textit{Uniform(0,1)}.

\section{Algoritmo}
L'algoritmo del test degli estremi effettua un raggruppamento (\textit{batching}) dei 
valori estratti dal generatore in gruppi di uguale lunghezza (determinato dal 
parametro \textit{d}), trovando il 
massimo di ogni batch, elevando tale massimo all d-esima potenza e conteggiando 
tutti i massimi generati in un array. Tale funzione viene applicata per ogni 
stream del generatore di Lehmer.
Tutto ci\`o \'e illustrato di seguito:
\begin{verbatim}
  for(stream = 0; stream < 256L; stream++) {
      long o[K];
      memset(o, 0, K * sizeof(long));
      for(i = 0; i < N; i++) {
          double r = Random();
          for(j = 1; j < D; j++) {
              u = Random();
              if(u > r) r = u;
          }
          u = exp(D * log(r));
          x = u * K;
          o[x] ++;
      }
  }
\end{verbatim}
Per ogni stream, si determina quindi la variabile chi-quadro $v$ tramite questa 
funzione:
\begin{verbatim}
      for(i = 0; i < K; i++) {
	      v += (square(o[i] - e_x));
      }
      v /= e_x;
\end{verbatim}

I valori critici $v_1^{*}$ e $v_2^{*}$ vengono calcolati utilizzando la funzione inversa 
\textit{idfChisquare(long n, double u)}  fornita dalla libreria rvms.c del libro di 
testo. Bisogna precisare che per il calcolo 
di tali variabili statistiche \'e stato scelto un livello di confidenza con 
parametro $\alpha$ = 0.05, mentre i parametri N e K sono rispettivamente N = 100000 
e K = N/20 = 5000.
Successivamente si confronta la statistica chi-quadro \textit{v}, determinata al passo 
precedente, con i valori critici  $v_1^{*}$ e $v_2^{*}$ , per ogni stream (in totale sono 256 
variabili chi-quadro $v$ ).
Se $v < v_1^{*}$ o $v > v_2^{*}$ il test fallisce (per quello stream) con probabilit\'a $1 - 
\alpha$.

\section{Test e conclusioni}
\noindent Il grafico risultante di questo test empirico \'e illustrato di seguito:

\begin{figure}[H]
 \begin{center}
  \includegraphics[scale=0.45]{img/test.png}
  \caption[Test degli estremi]{Test degli estremi}
  \end{center}
\end{figure}

\vspace{0.5cm}
I valori critici sono visualizzati come linee orizzontali: quella inferiore rappresenta
$v_1^{*}$ = 4804.92 mentre quella superiore è $v_2^{*}$ = 5196.86.

Dalla simulazione effettuata si \'e notato che il numero di test statistici $v > v_2^{*}$ sono stati esattamente 7, come il numero di test $v < v_1^{*}$.

\vspace{0.5cm}
Di seguito \'e riportato l'output del programma:

\begin{figure}[H]
  \begin{center}
  \includegraphics[scale=0.7]{img/test_estremi.png}
  \caption[Risultati]{Risultati}
  \end{center}
\end{figure}


Considerando il numero totale di test falliti (\textit{upper} e \textit{lower bound}) pari a 14, si nota che non
ci si discosta molto rispetto al valore atteso approssimato; infatti in 256 test con un livello di
confidenza del 95\% il valore atteso è circa 256 * 0.05 = 13 fallimenti. Questo valore pu\`o 
essere una indicazione della bont\`a del generatore di Lehmer implementato.
   	 % 1. Generatore di Lehemer
	\chapter{Obiettivi}
 	\label{cap:obiettivi}
 L'obiettivo del progetto è quello di modellare, pianificare e sviluppare un simulatore di traffico web che rispetti le specifiche di consegna.
 
\vspace{0.5 cm} \noindent Il sistema reale presenta le seguenti caratteristiche:
 \begin{itemize}
 \item un flusso di utenti che si connettono al sistema sotto forma di sessioni;
 \item un numero di richieste di cui si compone ogni sessione;
 \item un front-end server;
 \item un back-end server con un database.
 \end{itemize}

 Il simulatore verr\'a utilizzato per analizzare il comportamento stazionario relativo ad alcuni indici prestazionali quali il tempo di risposta del sistema, il throughput e la percentuale di sessioni abortite e rifiutate.
 Il sistema dispone infatti anche di un meccanismo di \emph{gestione del sovraccarico} basato sul monitoraggio in tempo reale dell'utilizzazione del front-end.\\

% Le specifiche fornite dalla traccia sono le seguenti:
 %\begin{itemize}
 %	\item $\lambda_{sessioni} = 35 \frac{richieste}{s}$ (distribuito esponenzialmente)
 %	\item $Dimensione_{sessioni} \sim Equilikely(5, 35)$
 %	\item $E[Z] = 7 s$ (distribuito esponenzialmente)
 %	\item $E[D]_{front-end} = 0.00456 s$ (distribuito esponenzialmente)
 %	\item $E[D]_{back-end} = 0.00117 s$ (distribuito esponenzialmente)
 %\end{itemize} 
	 % 2. Obiettivi
 	\def \ti{\textit}
\def \bf{\textbf}

\chapter{Generazione numeri pseudo-casuali}
	\label{cap:generatorerandom}

Per la generazione di numeri pseudo casuali, il simulatore implementato, utilizza il \textbf{Generatore di Lehmer} con le relative funzioni implementate all'interno delle librerie \textit{rngs} e \textit{rvms}. Nella fase di impostazione del simulatore (ovvero durante il lancio), viene chiesto all'utente di scegliere quale seed adoperare, in modo che il generatore possa elaborare numeri pseudo-casuali partendo da quest'ultimo.

\section{Generatore di Lehmer}
Il generatore di Lehmer è un generatore basato su un algoritmo che da origine ad una sequenza di numeri pseudo-casuali. \'E definito da due parametri:

\begin{itemize}
 \item un modulo m, che è un numero primo molto grande (in questo caso si è usato $2^{31} - 1$);
 \item un moltiplicatore a che rappresenta un numero intero compreso tra 1 ed m-1.
\end{itemize}

La sequenza numerica pseudo-random viene generata tramite la formula :

\begin{center} $x_{i+1} = ax_{i} mod m$ \end{center}

\noindent Questa sequenza parte da un numero $x_{0}$ detto seed, anch'esso scelto tra 1 ed m - 1. Non tutte le combinazioni di a ed m però sono ottimali per realizzare una sequenza di numeri che garantiscano una buona randomicità. Per vedere dunque se un seed e un moltiplicatore garantiscano un buon livello di randomicità esistono dei test empirici.

\section{Test degli estremi}
Il test scelto per verificare la correttezza dei numeri generati è quello conosciuto come ``\textit{Test degli estremi}''. Questo test si basa sul teorema seguente:

\vspace{0.5cm} \noindent \textbf{Teorema} Se $U_{0}^{}$, $U_{1}^{}$,...,$U_{d-1}^{}$ è una sequenza di variabili \textit{Uniform(0,1)} e se 

\begin{center}R = max{$U_{0}^{}$, $U_{1}^{}$,...,$U_{d-1}^{}$} \end{center} 

\noindent allora la variabile U = $R_{}^{d}$ è una \textit{Uniform(0,1)} \footnote{Attenzione: il teorema afferma che la variabile U è una Uniform(0,1), mentre la variabile R non lo è.}.

In pratica verifica che la variabilità delle altezze dell'istogramma prodotto dai numeri pseudo-casuali generati è sufficientemente piccola da poter concludere che questi numeri appartengano ad una popolazione \textit{Uniform(0,1)}. Questa operazione è fondamentale in quanto tutte le altre distribuzioni di probabilità contenute all'interno della libreria utilizzata vengono generati a partire dalla \textit{Uniform(0,1)}.

\section{Algoritmo}
Il comportamento del generatore può essere testato raggruppando gli output del generatore \textit{d} termini alla volta, trovando il massimo di ogni batch, elevando il massimo alla \textit{d}-esima potenza, e contando tutti i massimi così generati. Vediamo un esempio di questo algoritmo sotto forma di codice:

\begin{verbatim}
  for(stream = 0; stream < 256L; stream++) {
      long o[K];
      memset(o, 0, K * sizeof(long));
      for(i = 0; i < N; i++) {
          double r = Random();
          for(j = 1; j < D; j++) {
              u = Random();
              if(u > r) r = u;
          }
          u = exp(D * log(r));
          x = u * K;
          o[x] ++;
      }
  }
\end{verbatim}

Per effettuare correttamente questo test è necessario prendere $K >= 1000$, $N >= 10K$, $d >= 2$. Nel nostro caso abbiamo scelto $K = 5000$, $N = 100000$ e $d = 7$. Una volta selezionati i valori si procede all'esecuzione in questo modo:
\begin{itemize}
 \item Usare l'algoritmo precedentemente descritto per calcolare l'istogramma.
 \item Calcolare la statistica \textit{v} Chi-Quadro.
 \item Scegliere l'intervallo di confidenza che vogliamo verificare, nel nostro caso è al 95\%.
 \item Determinare le soglie critiche $v_{1}^{*}$ e $v_{2}^{*}$.
\end{itemize}
Se \textit{v $<$ $v_{1}^{*}$} o \textit{v $>$ $v_{2}^{*}$} il test è da considerarsi \textit{fallito}.

\section{Test e conclusioni}
\noindent Vediamo per prima cosa come si comporta il simulatore nel caso il seed sia 48271.

\begin{figure}[H]
  \centering
  \includegraphics[scale=0.55]{img/result_48271.png}
  \caption[Test degli estremi]{SEED 48271.}
  \label{fig:result_48271}
\end{figure}

\noindent Oltre al seed proposto il test è stato eseguito anche sui seguenti:
\begin{itemize}
\item 615425336
\begin{figure}[H]
  \centering
  \includegraphics[scale=0.55]{img/result_615425336.png}
  \caption[Test degli estremi]{SEED 615425336.}
  \label{fig:result_615425336}
\end{figure}
\item 37524306
\begin{figure}[H]
  \centering
  \includegraphics[scale=0.5]{img/result_37524306.png}
  \caption[Test degli estremi]{SEED 37524306.}
  \label{fig:result_37524306}
\end{figure}
\item 123456789
\begin{figure}[H]
  \centering
  \includegraphics[scale=0.5]{img/result_123456789.png}
  \caption[Test degli estremi]{SEED 123456789.}
  \label{fig:result_123456789}
\end{figure}
\end{itemize}

\vspace{0.5cm}
Alla luce dei diagrammi rappresentati si può vedere che su 256 test effettuati (uno per ogni stream del generatore), la maggior parte non è fallita sottolineando la buona randomicità del generatore, in quanto le statistiche rappresentate sono contenute all'interno dell'intervallo rappresentato dal massimo e dal minimo valore critico ($v_{i}^{*}$).

  	 % 3. Modello Concettuale
 	\chapter{Modello Progettuale}
	\label{cap:modelloprogettuale}

\section{Architettura}
Il diagramma del sistema da implementare pu\`o essere 
rappresentato semplicemente come in figura \ref{fig:architettura}
\begin{figure}[H]
	\centering
	\includegraphics[scale=0.7]{img/architettura.png}
	\caption[Architettura del sistema]{Rappresentazione grafica 
dell'architettura del sistema.}
	\label{fig:architettura}
	\end{figure}
All'inizio della simulazione il primo evento che si verifica \'e sempre l'arrivo 
di una nuova sessione (evento NewSession).

\vspace{0.5cm}I client del centro delle sessioni attive, all'interno del ramo di 
retroazione sono paralleli ed infiniti, definendo un infinity server.

\vspace{0.5cm}I nuovi eventi in arrivo verso il sistema verranno 
posti in coda al FrontServer, nel caso in cui non possano essere serviti. Come 
\'e visibile dalla figura precedente, la sessione una volta servita dal 
Front Server verr\`a posta in coda verso il Back-End Server finch\'e non verr\`a 
servita da quest'ultimo. Infine le sessioni saranno completate oppure poste 
nella zona di "\textit{Thinking}" finch\'e non verranno reinserite nella coda 
del Front-End in attesa di essere serviti, passando attraverso un ramo di 
feedback.
 
\section{Clock di simulazione e schedulazione di  Eventi}
Nella fase implementativa si tiene conto dell'avanzamento del tempo per mezzo 
della variabile \textit{current\_time}. Il meccanismo di avanzamento del tempo 
scelto \'e il \textit{Next-Event Time Advance}. Questa scelta garantisce che gli 
eventi occorrano nella sequenza corretta, ovvero vengono processati in ordine 
crescente rispetto al tempo di schedulazione. Si utilizza, inoltre, il flag di 
\textit{arrivals} per regolare l'accettazione delle nuove sessioni, necessaria per il meccanismo di overload management: se impostata 
a zero vengono inibiti i nuovi arrivi\footnote{Ovvero viene eseguito il drop 
della sessione in entrata o l'abort se la sessione \'e gi\`a nel sistema}, 
altrimenti si procede normalmente con la simulazione.

\section{Event List}
Per la gestione degli eventi si utilizza una lista collegata di strutture 
\textit{Event}, come quella mostrata in figura, salvate in ordine crescente 
rispetto al tempo. Ogni nodo contiene il tempo di occorrenza e la sua tipologia. 
Un gestore di eventi \'e utilizzato per il demultiplexing di tale lista facendo 
uso di una funzione pop() per ottenere il \textit{Next-Event} da processare.
\begin{figure}[H]
  \centering
  \includegraphics[scale=0.5]{img/EventList.png}
  \caption[EventList]{Struttura lista eventi}
  \label{fig:eventList}
\end{figure}

\section{Arrival Queue}
Al fine di ottenere informazioni riguardo i tempi di attesa che le sessioni 
sperimentano durante la loro permanenza nel sistema, si utilizzano delle 
strutture dati atte a registrare tali informazioni ed utilizzate negli algoritmi 
del calcolo delle medie. Tali strutture, denominate \textit{ArrivalQueue}, 
immagazzinano i tempi di arrivo delle sessioni nelle sottosezioni del sistema 
(ovvero quando una sessione entra nel \textit{Front Server} o nella sua coda, 
quando entra nel \textit{Back-End Server}, e cos\`i via). Ad ogni completamento 
sperimentato da una sessione viene utilizzata la coda relativa e si calcola, per 
differenza, il tempo effettivo che la sessione ha passato in quella parte di 
sistema. Tutto ci\`o \'e possibile grazie all'ipotesi che l'ordine di arrivo, 
all'interno delle code del sistema, \'e sempre \texttt{preservato}. Infatti la 
prima sessione ad entrare nella coda del Front Server, ad esempio, sar\`a la 
prima a lasciarlo.

\section{Request Queue}
Dal momento che \'e impossibile identificare una singola richiesta utilizzando 
la Next-Event Simulation, il problema di preservare l'informazione riguardante 
il numero di richieste attive di cui si compone una sessione viene risolto con 
la struttura dati \textit{Request Queue}. 

Ad ogni richiesta completata si decrementa il contatore delle richieste relative 
a quella sessione in modo da propagare tale informazione a tutte le richieste 
future. Quando il contatore arriva a zero la sessione viene completata del tutto 
e di conseguenza abbandona il sistema.

\section{Client Order List}
Per preservare una Next-Event Simulation priva di contaminazioni derivanti dall'
aggiunta di dati identificanti le sessione o le richieste all'interno degli eventi di base, 
la Client Order List permette di conservare le 
informazioni relative all'ordine di arrivo e di uscita degli utenti all'interno 
del centro Client. Attraverso questa informazione \'e possibile gestire il 
corretto ordine degli elementi della Request Queue nonostante siano condizionati 
da una mancanza di determinismo circa l'ordine di completamento degli utenti 
durante il loro periodo di Think Time.


\section{Personalizzazione del modello}
In base alle scelte effettuate dall'utente nella fase di \textit{setup} \'e 
possibile decidere di avviare la simulazione:
\begin{itemize}
\item senza \textbf{Overload Management}
\item con \textbf{Overload Management}
\end{itemize}
 
\noindent \'E inoltre possibile scegliere quale distribuzione utilizzare tra:
\begin{itemize}
\item \textit{Esponenziale}
\item \textit{10-Erlang}
\item \textit{Iperesponenziale}
\end{itemize}

\noindent Infinie si devono impostare i parametri riguardanti il \textit{numero di batch}
e la \textit{grandezza del batch} stesso.

%%Diagrammi per spiegazioni
\section{Algoritmi di Gestioni eventi}
\subsection{NewSession}
\begin{figure}[H]
  \centering
  \includegraphics[scale=0.35]{img/NewSession.png}
  %\caption[NewSession]{Struttura lista eventi}
  \label{fig:NewSession}
\end{figure}
\subsection{FS Completion}
\begin{figure}[H]
  \centering
  \includegraphics[scale=0.35]{img/FS_Completion.png}
  %\caption[NewSession]{Struttura lista eventi}
  \label{fig:FS_Completion}
\end{figure}

\subsection{BES Completion}
\begin{figure}[H]
  \centering
  \includegraphics[scale=0.40]{img/BES_Completion.png}
  %\caption[NewSession]{Struttura lista eventi}
  \label{fig:FS_Completion}
\end{figure}

\subsection{Client Completion}
\begin{figure}[H]
  \centering
  \includegraphics[scale=0.45]{img/CLIENT_Completion.png}
  %\caption[NewSession]{Struttura lista eventi}
  \label{fig:FS_Completion}
\end{figure}

\section{Indici di prestazioni}
Il simulatore qui utilizzato genera un insieme di statistiche che permettono di ricavare 
informazioni utili per comprendere il comportamento del sistema.
\vspace{0.5cm}
\noindent Gli indici di prestazione calcolati sono:
\begin{itemize}
\item \textbf{Useful Troughput}:
indica il totale delle sessioni completate nell'unit\`a di tempo. 
Tale indice viene calcolato attraverso il rapporto tra le sessioni totali 
processate dal sistema e l'intervallo di tempo necessario per ultimare questo 
compito. Il troughput basato sulle sessioni \'e un ottimo indice per valutare 
il numero di utenti serviti dal sistema in un intervallo dato, risultando una 
misura sensibile per l'utente finale.

\item \textbf{Tempo di Risposta del Sistema}:
viene inteso come la somma del tempo di risposta: Tempo in coda + Tempo di 
Servizio.
In pratica il tempo di risposta \'e inteso come il tempo che intercorre tra l'istante in cui una
richiesta entra nel Front Server e l'uscita della stessa dal Back-End Server.

\item \textbf{Drop Ratio}:
misura il rapporto tra il totale delle sessioni rifiutate dal sistema ed il 
numero di sessioni totali che tentano di entrare nel sistema (accettate + 
rifiutate).

\item \textbf{Abort Ratio}:
\'e il rapporto tra le richieste abortite ed il totale di richieste processate 
dal sistema.
Questo \'e dovuto al fatto che una richiesta pu\`o rientrare nel sistema, 
tuttavia in condizioni di saturazione, tale richiesta non \`e in grado di poter 
rientrare all'interno del Front Server, quindi viene appunto abortita.

\end{itemize}
	 % 4. Modello Progettuale
	\chapter{Modello Computazionale}
Il programma realizzato \'e composto da un file eseguibile (simulatore.c) e di alcuni file dove sono contenute funzioni di appoggio. (\textit{event\_list}, \textit{arrival\_queue}, \textit{autocorrelation}, \textit{client\_req}, \textit{req\_queue}).
Il software sviluppato \'e codificato con il linguaggio \textit{C}.

\section{Simulatore.c}

Questo applicativo \'e il cuore del simulatore implementato. Tale programma sfrutta un'interfaccia testuale per interrogare l'utente circa la configurazione da adottare per la simulazione da eseguire. 
\'E possibile selezionare diverse opzioni:
\begin{itemize}
\item Scegliere la distribuzione da testare tra:
\begin{itemize}
\item \textit{Esponenziale}
\item \textit{HyperEsponenziale}
\item \textit{10-Erlang}
\end{itemize}
\item Scegliere il seed per la generazione di numeri random
\begin{itemize}
\item 615425336
\item 37524306
\item 123456789
\end{itemize}
\item Scegliere i parametri della simulazione:
\begin{itemize}
\item \textit{Primo Stop}
\item \textit{Ultimo Stop}
\item \textit{Numero di esecuzioni}
\end{itemize}
\item La possibilit\'a di stampare sulla console le variabili più importanti durante la simulazione.
\end{itemize}

Alla luce di quanto illustrato in precedenza si pu\'o sottolineare la capacit\'a del programma di poter calcolare automaticamente il passo della simulazione: l'utente dovr\'a soltanto inserire il parametro iniziale e finale della simulazione ed il numero di run che desidera eseguire, il software generer\'a il passo unitario per il ciclo di simulazione.
Inoltre all'interno del programma \'e stato implementato un meccanismo di \textit{Overload Management}, che l'utente pu\'o abilitare o disabilitare, applicato poi nella distribuzione della \textit{10-Erlang}.
I risultati prodotti da questa simulazione vengono trascritti su un file di tipo "\textit{.csv}"  in modo da dare all'utente una chiara ed equilibrata visione dei dati ottenuti.

\section{Event List}
La lista di eventi \'e costituita da strutture di tipo \textit{Event}, formate da un campo di tipo \textit{double}, che indica il time che rappresenta il tempo di occorrenza, un \textit{\_EVENT\_TYPE} type rappresentante il tipo di evento ed un puntatore \textit{next} alla struttura seguente.
Per merito della funzione \textit{add\_event()} \'e possibile aggiungere eventi alla lista. Verranno inseriti seguendo un ordinamento crescente rispetto alla variabile time. Durante l'inserimento dei dati viene effettuato un controllo sulla consistenza dei dati, cioè: si controlla, con una funzione \textit{event\_check()}, che il tempo sia un valore positivo, che il tipo di evento sia compreso tra 0 e 3.
Gli eventi vengono estratti dalla struttura attraverso la funzione \textit{pop\_event()}, che preleva l'evento in testa alla lista, restituendolo alla funzione chiamante.

\section{Arrival queue}
La coda di arrivi contiene la struttura di dati di riferimento, che permette la gestione degli arrivi nei vari centri quali: \textit{Front Server}, \textit{Back-End Server}, \textit{Centro Client}. La coda viene rappresentata con una semplice struttura dati formata da: un tempo di arrivo e un puntatore all'elemento successivo. Le funzioni messe a disposizione per questa struttura di dati sono: \textit{arrival\_add()}, \textit{arrival\_pop()} e l'\textit{arrival\_print()}.
\begin{itemize}
\item \textit{arrival\_add()}: genera un nuovo elemento contenente il tempo di arrivo in un determinato centro,scorre tutto il contenuto della lista generata, posizionando il nuovo nodo in fondo alla coda.
\item \textit{arrival\_pop()}: permette l'estrazione dell'elemento meno recente della coda.
\item \textit{arrival\_print()}: stampa lo stato della coda.
\end{itemize}

\section{Request queue}
La coda delle richieste si basa sul concetto di richiesta: ogni sessione, una volta ammessa all'interno del sistema, definisce un numero di richieste compreso tra 5 e 35. Questa informazione viene inserita all'interno della coda delle richieste in modo da poter sfruttare i dati generati, per modellare gli utenti attivi durante la simulazione.
Sfruttando la funzione \textit{enqueue\_req()} \'e possibile inserire tutte le richieste generate dalla nuova sessione vigente. Per poter rimuovere tale dato \'e possibile  utilizzare la \textit{dequeue\_req()}.
Per verificare l'andamento di tale struttura dati \'e stata implementata la funzione \textit{print\_req()}.

\section{Client request}
La client\_req.h implementa una struttura dati in grado di propagare l'informazione circa il numero di richieste relative ad una sessione. Questa viene impiegata all'ingresso e all'uscita dal centro di client in quanto l'ordine di entrata all'interno della zona del \textit{Think Time} \'e differente da quello di uscita.
Per quanto riguarda le funzioni, implementate all'interno di questo programma hanno le stesse funzionalit\'a espresse nelle sezioni precedenti, sono: \textit{add\_client\_req()}, \textit{pop\_ClientReq()}, \textit{print\_client\_req()}.

\section{File Manager}
La parte relativa al file manager gestisce tutto il flusso di dati da trascrivere  su un file di formato "\textit{.csv}". Composto da tre funzioni:
\begin{itemize}
\item \textit{get\_date()}: permette di ottenere l'orario e la data correnti, da salvare sul file desiderato.
\item \textit{open\_file()}: funzione utilizzata per la creazione e l'apertura del file da salvare. Il nome del file viene elaborato utilizzato la funzione \textbf{get\_date()}, quindi con la data corrente della creazione più il tipo di distribuzione scelta durante la fase di setting.
\item \textit{close\_file()}: funzione adottata per chiudere il file una volta terminata la scrittura su di esso.
\end{itemize}

\section{Utils}
Il file utils.h contiene delle funzioni utilizzate per la pulizia della console e alcune funzioni di scrittura su file di tipo \textit{.csv}, circa i dati ottenuti al termine di una simulazione completata.

\section{Salvataggio dei dati}
Il programma "\textbf{simulatore}" serializza tutta le informazioni, generate durante la simulazione, utili al calcolo delle grandezze medie su un file, inserendo blocchi di dati al termine di ogni run.
Il nome dei file \'e legato alla data ed all'orario di avvio della simulazione e al tipo di distribuzione adottata.
Il tipo di file che viene generato \'e di tipo \textit{.xls}. Questi sono adibiti alla creazione automatica dei grafici. Il programma quindi generata automaticamente un file con delimitatori di cella e di riga al fine di consentire all'utente una facile consultazione dei dati ottenuti o semplicemente una facile creazione dei diagrammi. Questi file verranno adottati durante la fase di analisi dei risultati.

\section{Autocorrelazione}
Al fine di calcolare l'autocorrelazione sui tempi di risposta del sistema con \textbf{LAG} pari a 20 \'e stato scritto un programma C che prende in input il file generato dal simulatore e restituisce i valori calcolati delle autocorrelazioni. Per il calcolo di queste si \'e usata la formula:

\vspace{0.5cm}
\begin{center} $r_{j} = \frac{c_{j}}{c_{0}} con j = 1,2,...20$\end{center}

\section{Intervalli di confidenza}
\'E stato sviluppato un programma scritto in linguaggio \textbf{C} allo scopo di calcolare gli intervalli di confidenza di livello 1 - \textalpha, dove \textalpha è pari al 5\%. Tale programma prende in input un file in cui sono scritti tutti i valori su cui si vuole fare il calcolo e computa le seguenti statistiche:
\begin{itemize}
 \item Calcolo della media: $\bar{x}_{i} = \frac{1}{i}(x_{i} - \bar{x}_{i-1})$;
 \item Calcolo della varianza: $v_{i} = v_{i-1} + (\frac{i-1}{i}){(x_{i} - \bar{x}_{i-1})}^{2}$;
 \item Calcolo del valore critico: $t^{*} = idfStudent(n-1, 1-\frac{\textalpha}{2})$;
 \item Calcolo degli estremi dell'intervallo: $\bar{x} \pm \frac{t^{*}s}{\sqrt{n-1}}$;
\end{itemize}     % 5. Modello Computazionale
 	\chapter{Verifica}

La fase di verifica \'e molto importante poich\'e consente di dimostrare la 
consistenza del programma creato con il modello delle specifiche.

\vspace{0.3cm}In primis, \'e stata utilizzata una funzione di stampa per 
verificare il corretto flusso delle sessioni all'interno dell'intero sistema. 
Tale funzione, la \texttt{print\_system\_state()}, stampa le statistiche pi\`u 
importanti del sistema in tempo reale su standard output.

\vspace{0.3cm}In secundis, si \'e notato e verificato che le liste contenenti le 
sessioni e le richieste si riempiono e si svuotano in modo corretto. Anche in 
questi casi sono state necessarie funzioni di stampa per verificare che 
l'aggiornamento fosse adeguato.

\vspace{0.3cm}In terzis, i vincoli sullo stato del sistema e sull'entrata in 
azione dell'overload manager, e sul calcolo delle medie sono tutti soddisfatti.

\vspace{0.3cm}Il simulatore parte con il numero di sessioni nullo e tutte le 
variabili di stato e di supporto sono inizializzate opportunamente. Il 
sistema consente di avere una stampa aggiornata di tutti i parametri 
rilevanti, come throughput e tempo medio di risposta.

\vspace{0.3cm}Il numero di sessioni rifiutate e abortite cresce in maniera 
consistente con il meccanismo di controllo delle ammissioni.
\begin{comment}
Nel caso in cui l'utente non utilizzi l'overload management, \'e stato dimostrato che superato il tempo di
STOP finale, chiamato FIN all'interno del codice, nessuna nuova sessione
sarebbe stata accettata dal sistema e, successivamente la simulazione
sarebbe stata interrotta perch\'e la coda tenderebbe a crescere all'infinito.
\end{comment}
\vspace{0.3cm}Infine, come ultima verifica, \'e stato dimostrato che superato il 
numero massimo di sessioni per batch, che l'utente ha inserito manualmente, il sistema passa a simulare il batch
successivo, e una volta raggiunto il numero massimo di batch, impostati sempre dall'utente, il programma termina
correttamente la propria esecuzione.

     % 6. Verifica
 	\chapter{Validazione} 

In questa fase viene testata la corrispondenza del simulatore con il modello reale. Sono stati effettuati i seguenti controlli:

\begin{itemize}
 \item Saturazione del Front Server: ci sono due concause che portano alla saturazione del Front Server, in primo luogo c'\'e una enorme differenza tra il tasso di servizio di quest'ultimo e quello del Back-End Server che \'e circa 3 volte maggiore; inoltre il numero di utenti che stazionano nel centro Client tende a crescere enormemente e tale comportamento influenza l'andamento di richieste in entrata al Front Server che ne determina la congestione.
 
 \item La coda del Back-End Server risulta essere quasi sempre vuota poich\'e \'e il Front Server che elabora in maniera lenta mentre il tasso di servizio in questo caso \'e di oltre 800 richieste / secondo; questo non permette la creazione di coda e di conseguenza l'utilizzazione \'e molto bassa.
 
 \item Il numero di client \'e notevolmente elevato: questo \'e dovuto all'elevato Think Time sperimentato dagli utenti, circa 7 secondi a client. Considerando il throughput del sistema ed i tassi di servizio dei due serventi \'e immediato notare come il numero dei client attivi contemporaneamente tenda a crescere nel tempo. Tale componente è modellato come un Infinite Server.
 
 \item Nel caso di overload management la percentuale delle connessioni rifiutate e abortite \'e molto alta arrivando a toccare anche l'87\% nel secondo caso, mentre nel primo \'e pi\'u contenuta, con un massimo che si attesta circa attorno al 30\% per poi oscillare in quell'intorno. Questo perch\'e il Front Server rappresenta il collo di bottiglia dell'intero sistema.
\end{itemize}

\section{Modello Analitico}
Al fine di prevedere, in linea di massima, i risultati del simulatore, viene 
elaborato un modello  analitico semplificato per poter studiare il sistema preso 
in esame.
Il sistema implementato presenta numerosi vincoli ed una complessit\'a insita 
nelle specifiche, a tal proposito viene proposto un modello volontariamente e 
lievemente differente dal caso reale, fornendo un limite inferiore delle 
prestazioni e degli indici misurati.

Il sistema esaminato risulta essere interattivo, in quanto avviene uno scambio 
di richieste tra Client e Server, con un comportamento a rete aperta ( le nuove 
sessioni possono entrare nel sistema qualora questo non fosse saturo), a tal 
proposito si \'e deciso di presentare due modelli differenti tra loro, uno a 
rete aperta ed uno a rete chiusa. Questi permettono di descrivere in dettaglio 
la maggior parte degli elementi che costituiscono il sistema stesso.

\section{Modello semplificato a rete aperta}
In questo primo modello si ha una rete aperta di Jackson con un tasso di 
sessioni in
arrivo pari a $\gamma$ . Si \'e deciso cos\'i, per questioni di semplificazione 
del modello, di non
inserire, come componenti, la Priority Queue e il centro di Client, in quanto
risultavano di difficile analisi in un modello di rete aperta come quello di 
Jackson.
\begin{center}	
	\begin{figure}[H]
	\centering
	\includegraphics[scale=0.7]{img/reteJackson.png}
	\caption[Modello a rete aperta]{Modello a rete aperta}
	\label{fig:Modello a rete aperta}
	\end{figure}
\end{center}

$\gamma = 35 sessioni/s$

$\vspace{2mm}$

$\begin{cases} 
\lambda_{1} = \gamma + p \lambda_{2} \\ \lambda_{2} = \lambda_{1} \\
\end{cases}$  $\rightarrow$
$\begin{cases} 
\lambda_{1}(1-p) = \gamma \\ \lambda_{2} = \lambda_{1} \\
\end{cases}$ $\rightarrow$
$\begin{cases} 
\lambda_{1} =\frac{ \gamma}{(1- p)} \\ \lambda_{2} =\frac{\lambda_{2}}{(1-p)} \\
\end{cases}$
$\vspace{2mm}$
$\lambda_{1} = \lambda_{2} = \frac{\gamma}{(1-p)} ; p=\frac{19}{20} \rightarrow 
\lambda_{1} = \lambda_{2} = 35\times20 = 700 richieste/s$
$\vspace{2mm}$

$\lambda_{1}\gg\mu_{FS}; \lambda_{2}=\mu_{FS} poichè \mu_{BES}\gg\lambda_{1}$
$\vspace{2mm}$
 Viene quindi calcolato il throughput, ovvero il numero di sessioni che escono 
dal sistema al secondo:
$\vspace{2mm}$

$\lambda_{2} =\frac{\lambda_{2}}{(1-p)}=\frac{\mu_{FS}}{20}\approx\frac{219}{20} 
richieste/s =X_{sessioni}$
$\vspace{2mm}$
$\vspace{2mm}$
$X_{richieste}=X_{sessioni}\times20$
$\vspace{2mm}$
Per quanto concerne l'indice riguardante la percentuale di sessioni rifiutate 
dal sistema, si trova:
$\vspace{2mm}$

$dropped=\frac{\#sessioni accettate}{\#totale arrivi}$ $\rightarrow$ 
$\frac{35-X_{sessioni}}{35} = 1-\frac{2}{7}\approx 0.7$ $\rightarrow$ $\%dropped 
\approx 70\%$

\section{Modello semplificato chiuso}
\begin{center}	
	\begin{figure}[H]
	\centering
	\includegraphics[scale=0.7]{img/retechiusa.png}
	\caption[Modello a rete chiusa]{Modello a rete chiusa}
	\label{fig:Modello a rete aperta}
	\end{figure}
\end{center}

$D_{FS}=0.00456; D_{BES}=0.00117; E[Z]=7s ; N=250$ (in realt\'a il numero 
aumenta costantemente!).
$\vspace{2cm}\hspace{2mm}$
$\vspace{2mm}\hspace{2mm}$ Anche qui si pu\'o calcolare il throughput del 
sistema, come:
$\vspace{2mm}$
$X = min\{\frac{1}{D_{max}},\frac{N}{D+Z}\}= 
min\{\frac{1}{D_{FS}},\frac{250}{D_{FS}+D_{BES}+Z}\}=35.685 richieste/s$.
$\vspace{2mm}$
Da cui si pu\'o ricavare il lower bound per il tempo medio di risposta.
$\vspace{2mm}$
$E[R]\geq max\{D,\frac{N}{X}-E[Z]\}= max 
\{0.00573,\frac{250}{35}-7\}\approx0.0057447$

     % 7. Validazione
 	\chapter{Progettazione degli esperimenti} 

\noindent Gli esperimenti eseguiti sono stati i seguenti:

\begin{enumerate}
 \item Simulazione del sistema quando il Front Server ha distribuzione esponenziale;
 \item Simulazione del sistema quando il Front Server ha distribuzione 10-Erlang;
 \item Simulazione del sistema quando il Front Server ha distribuzione Iperesponenziale;
 \item Simulazione del sistema quando il Front Server ha distribuzione 10-Erlang con soglia di utilizzazione pari all'85\%.
\end{enumerate}

\noindent \vspace{0.5cm} Gli esperimenti sono stati basati sui seguenti parametri:
\begin{itemize}
 \item SEED : indica il seed scelto dall'utente nella fase di setup;
 \item TIME : durata della simulazione, ovvero lo STOP finale;
 \item STEP : indica ogni quanto tempo vengono calcolate le statistiche;
 \item THRESHOLD : indica se l'overload manager \'e attivo.
\end{itemize}

\begin{table}[H]
 \centering
 \begin{tabular}{|c|c|c|c|}
 \hline
 SEED & TIME & STEP & THRESHOLD\\ \hline
 1 & 5 & 7 & 9 \\ \hline
 1 & 5 & 7 & 9 \\ \hline
 1 & 5 & 7 & 9 \\ \hline
 \end{tabular}
\end{table}


Per Ogni simulazione si \'e scelto di salvare su file i parametri necessari all'elaborazione dei risultati ogni 100 secondi, per valutare quando il sistema entra in uno stato stazionario e per vedere l'evolversi della situazione nel tempo. Al superamento dell'istante di STOP la simulazione viene interrotta, senza attendere che la coda risulti vuota poich\'e crescendo esponenzialmente i tempi per farla svuotare sarebbero talmente alti da rendere la simulazione non pi\'u reale.     % 8. Progettazione Esperimenti e Simulazioni
 	\chapter{Analisi dei Risultati} 

I dati relativi agli indici di prestazione di interesse (tempi di risposta e throughput) misurati 
nelle precedenti fasi di test sono stati in seguito aggregati e visualizzati in forma di grafici. 
Infatti, in base al tipo di test effettuato, tali grafici si suddividono in due categorie: i grafici di 
throughput e tempi di risposta del sistema in stato di instabilit\`a (front server sovraccarico) e 
quelli relativi al sistema in condizione di stazionariet\`a (front server non sovraccarico), ovvero quelli gesiti
attraverso il meccanismo di overload management.

\section{Sistema senza Overload Management}

\subsection{Response Time}

In questo scenario l’utilizzazione del front-end server \'e praticamente uguale a 1, perci\`o 
non riesce a completare tutte le richieste entranti, con la inevitabile 
situazione di veder aumentare indefinitamente la lunghezza della sua coda. Di 
conseguenza il tempo di risposta del sistema tende a divergere all’aumentare del tempo di 
simulazione. Il grafico illustrato di seguito mostra tale scenario:

\begin{figure}[H]
 \centering
% \includegraphics[scale=0.7]{img/}
 \caption[Tempi di risposta del sistema instabile]{Tempi di risposta del sistema instabile}
 \label{fig:Tempi di risposta del sistema instabile}
\end{figure}


\subsection{Autocorrelazione}

\subsection{Useful Throughtput}

\section{Sistema con Overload Management}

\subsection{Response Time}

\subsection{Autocorrelazione}

\subsection{Useful Throughtput}

\subsection{Drop e Aborted Ratio}    % 10. Analisi dei Risultati

 	\appendix
 	\chapter{Codice}
	\label{appendix:codice}

\definecolor{dkgreen}{rgb}{0,0.6,0}
\definecolor{gray}{rgb}{0.5,0.5,0.5}
\definecolor{mauve}{rgb}{0.58,0,0.82}

\lstset{frame=tb,
  language=C,
  aboveskip=3mm,
  belowskip=3mm,
  showstringspaces=false,
  columns=flexible,
  basicstyle={\scriptsize\ttfamily},
  numbers=none,
  numberstyle=\scriptsize\color{gray},
  keywordstyle=\color{blue},
  commentstyle=\color{dkgreen},
  stringstyle=\color{mauve},
  breaklines=true,
  breakatwhitespace=true,
  tabsize=2
}

\section{Test degli estremi}
\subsection{test.c}
\lstinputlisting[breaklines]{../lehmer/test.c}

\section{Intervalli di Confidenza}
\subsection{interval\_calculator.h}
\lstinputlisting[breaklines]{../interval_calculator.h}

\section{Autocorrelazione}
\subsection{autocorrelation.h}
\lstinputlisting[breaklines]{../autocorrelation.h}

\section{Simulatore}
\subsection{simulatore.c}
\lstinputlisting[breaklines]{../simulatore.c}

\subsection{arrival\_queue.h}
\lstinputlisting[breaklines]{../arrival_queue.h}

\subsection{client\_req.h}
\lstinputlisting[breaklines]{../client_req.h}

\subsection{event\_list.h}
\lstinputlisting[breaklines]{../event_list.h}

\subsection{event\_manager.c}
\lstinputlisting[breaklines]{../event_manager.c}

\subsection{file\_manager.c}
\lstinputlisting[breaklines]{../file_manager.c}

\subsection{generate\_random\_value.h}
\lstinputlisting[breaklines]{../generate_random_value.h}

\subsection{global\_variables.h}
\lstinputlisting[breaklines]{../global_variables.h}

\subsection{req\_queue.h}
\lstinputlisting[breaklines]{../req_queue.h}

\subsection{rng.c}
\lstinputlisting[breaklines]{../rng.c}

\subsection{rng.h}
\lstinputlisting[breaklines]{../rng.h}

\subsection{rvms.c}
\lstinputlisting[breaklines]{../rvms.c}

\subsection{rvms.h}
\lstinputlisting[breaklines]{../rvms.h}

\subsection{simulation\_type.h}
\lstinputlisting[breaklines]{../simulation_type.h}

\subsection{user\_signal.c}
\lstinputlisting[breaklines]{../user_signal.c}

\subsection{utils.h}
\lstinputlisting[breaklines]{../utils.h}
      % Codice sorgente
 \end{document} 
