\chapter{Progettazione degli esperimenti} 

\noindent Gli esperimenti eseguiti sono stati i seguenti:

\begin{enumerate}
 \item Simulazione del sistema quando il Front Server ha distribuzione esponenziale;
 \item Simulazione del sistema quando il Front Server ha distribuzione 10-Erlang;
 \item Simulazione del sistema quando il Front Server ha distribuzione Iperesponenziale;
 \item Simulazione del sistema quando il Front Server ha distribuzione 10-Erlang con soglia di utilizzazione pari all'85\%.
\end{enumerate}

\noindent \vspace{0.5cm} Gli esperimenti sono stati basati sui seguenti parametri:
\begin{itemize}
 \item SEED : indica il seed scelto dall'utente nella fase di setup;
 \item TIME : durata della simulazione, ovvero lo STOP finale;
 \item STEP : indica ogni quanto tempo vengono calcolate le statistiche;
 \item THRESHOLD : indica se l'overload manager \'e attivo.
\end{itemize}

\begin{table}[H]
 \centering
 \begin{tabular}{|c|c|c|c|}
 \hline
 SEED & TIME & STEP & THRESHOLD\\ \hline
 1 & 5 & 7 & 9 \\ \hline
 1 & 5 & 7 & 9 \\ \hline
 1 & 5 & 7 & 9 \\ \hline
 \end{tabular}
\end{table}


Per Ogni simulazione si \'e scelto di salvare su file i parametri necessari all'elaborazione dei risultati ogni 100 secondi, per valutare quando il sistema entra in uno stato stazionario e per vedere l'evolversi della situazione nel tempo. Al superamento dell'istante di STOP la simulazione viene interrotta, senza attendere che la coda risulti vuota poich\'e crescendo esponenzialmente i tempi per farla svuotare sarebbero talmente alti da rendere la simulazione non pi\'u reale.