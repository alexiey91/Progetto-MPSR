\chapter{Progettazione Esperimenti e Simulazioni} 

Tutte le simulazione sono state effettuate su diversi computer, sia Pc desktop 
che notebook, sfruttando la portabilit\`a del software in modo tale da generare 
i dati sia su sistemi operativi \textit{Windows} che \textit{Linux}. La 
simulazione viene testata con parametri inseriti dall'utente manualmente, 
permettendo una facile impostazione del programma.
Sono state sfruttate macchine aventi processori multi-core al fine di eseguire 
pi\`u esecuzioni in parallelo.

\noindent Gli esperimenti eseguiti sono stati i seguenti:

\begin{enumerate}
 \item Simulazione del sistema quando il Front Server ha distribuzione 
esponenziale;
 \item Simulazione del sistema quando il Front Server ha distribuzione 
10-Erlang;
 \item Simulazione del sistema quando il Front Server ha distribuzione 
Iperesponenziale;
 \item Simulazione del sistema nel caso peggiore tra i precedenti con overload management attivo.
\end{enumerate}

\noindent \vspace{0.5cm} Gli esperimenti sono stati basati sui seguenti 
parametri:
\begin{itemize}
 \item \textbf{\textit{Seed}} : indica il seed scelto dall'utente nella fase di setup;
\begin{comment}
\item \textbf{\textit{START}} : indica l'inizio della simulazione, ovvero il primo stop.
\item \textbf{\textit{STOP}}  : indica il termine ultimo della simulazione.
\end{comment} 
 \item \textbf{\textit{Batch size}} : indica quante sessioni compongono un batch;
 \item \textbf{\textit{Batch number}} : indica quanti batch devono essere eseguiti in totale;
 \item \textbf{\textit{Threshold}} : indica se l'overload management \'e attivo.
 \begin{comment}
 \subitem \textbf{\textit{Batch size}} : indica quante sessioni compongono un batch;
 \subitem \textbf{\textit{Batch number}} : indica quanti batch devono essere eseguiti in totale;
 \end{comment}
\end{itemize}

Per ogni simulazione,che sfrutta l'overload management si \'e scelto di impostare la grandezza del \textit{batch} 
ed il suo numero di esecuzioni con lo stesso valore in modo da poter effettuare
un miglior confronto tra i diversi casi. Al termine di ogni \textit{batch} i dati 
ottenuti dalle metriche considerate vengono salvati su un file ``\textit{csv}''.
Al termine di tutti i \textit{bach} vengono computati e salvate su file ulteriori 
informazioni relative all'esecuzione generale, come ad esempio l'autocorrelazione o
gli intervalli di confidenza.
\begin{comment}
Per quanto riguarda le altre simulazioni non filtrate si è deciso di inserire lo stesso valore per lo \textit{START}, tra tutte le distribuzioni, ed analogamente per lo \textit{STOP}.
\end{comment}